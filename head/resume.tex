\chapter*{R\'esum\'e} % (fold)
\label{cha:resume}


Cette thèse développe des méthodes d'optimisation pour la gestion de la Supply Chain et a pour thème central la flexibilité définie comme la capacité à fournir un service ou un produit au consommateur dans un environnement incertain.
La recherche a été menée dans le cadre d'une convention CIFRE entre Argon Consulting, une société indépendante de conseil en Supply Chain, et l'\'Ecole des Ponts ParisTech.
Dans cette thèse, nous étudions trois sujets rencontrés par Argon Consulting et ses clients et qui correspondent à trois différents niveaux de décision (long terme, moyen terme et court terme).


Lorsque les entreprises élargissent leur portefeuille de produits, elles doivent décider dans quelles usines produire chaque article.
Il s'agit d'une décision à long terme, car une fois qu'elle est prise, elle ne peut être facilement modifiée.
Plus qu'un problème d'affectation où un article est produit par une seule usine, ce problème consiste à décider si certains articles doivent être produits par plusieurs usines et par lesquelles.
Cette interrogation est motivée par la grande incertitude de la demande.
En effet, pour satisfaire la demande, l'affectation doit pouvoir équilibrer la charge de travail entre les usines.
Nous appelons ce problème le multi-sourcing de la production.
Comme il ne s'agit pas d'un problème récurrent, il est essentiel de tenir compte du risque au moment de décider le niveau de multi-sourcing.
Nous proposons un modèle générique qui inclut les contraintes techniques du problème et une contrainte d'aversion au risque basée sur des mesures de risque issues de la théorie financière.
Nous développons un algorithme et une heuristique basés sur les outils standard de la Recherche Opérationnelle et de l'Optimisation Stochastique pour résoudre le problème du multi-sourcing et nous testons leur efficacité sur des données réelles.


Avant de planifier la production, certains indicateurs macroscopiques doivent être décidés à moyen terme tels la quantité de matières premières à commander ou la taille des lots produits.
Certaines entreprises utilisent des modèles de stock en temps continu, mais ces modèles reposent souvent sur un compromis entre les coûts de stock et les coûts de lancement.
Ces derniers sont des coûts fixes payés au lancement de la production et sont difficiles à estimer en pratique.
En revanche, à moyen terme, la flexibilité des moyens de production est déjà fixée et les entreprises estiment facilement le nombre maximal de lancements.
Poussés par cette observation, nous proposons des extensions de certains modèles classiques de gestion de stock en temps continu, sans coût de lancement et avec une limite sur le nombre de lancements.
Nous avons utilisé les outils standard de l'Optimisation Continue pour calculer les indicateurs macroscopiques optimaux.


Enfin, la planification de la production est une décision à court terme qui consiste à décider quels articles doivent être produits par la ligne de production pendant la période en cours.
Ce problème appartient à la classe bien étudiée des problèmes de Lot-Sizing.
Comme pour les décisions à moyen terme, ces problèmes reposent souvent sur un compromis entre les coûts de stock et les coûts de lancement.
Fondant notre modèle sur ces considérations industrielles, nous gardons le même point de vue (aucun coût de lancement et une borne supérieure sur le nombre de lancements) et proposons un nouveau modèle.
Bien qu'il s'agisse de décisions à court terme, les décisions de production doivent tenir compte de la demande future, qui demeure incertaine.
Nous résolvons notre problème de planification de la production à l'aide d'outils standard de Recherche Opérationnelle et d'Optimisation Stochastique, nous testons l'efficacité sur des données réelles et nous la comparons aux heuristiques utilisées par les clients d'Argon Consulting.



\secondcolor{Mots-clés} : Heuristique, Lot-Sizing, Mesure de Risque, Optimisation Stochastique, Problème d'affectation, Recherche Opérationnelle, Supply Chain Management.
