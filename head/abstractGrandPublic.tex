This thesis develops algorithms for rail transportation, and is conducted in relationship with Eurotunnel which operates the tunnel under the Channel.
It focuses on scheduling the trains in the tunnel and the objective is to satisfy operation constraints and commercial agreements with their partners. It also takes into account the delays to limit their propagation in the network.
The thesis also deals with the prices: we apply standard pricing methods to our scheduling algorithms to optimize in a global way the prices and the departure times. 
We focus in a third part on theoretical transportation problems, which consist in scheduling the departures of shuttles on a private network, to transport passengers, arriving continuously at an initial station, to a given destination. The shuttles are potentially allowed to perform several rotations. The objective is to minimize the waiting time of the passengers before the depart of their shuttle.