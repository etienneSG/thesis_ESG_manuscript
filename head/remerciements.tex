\chapter*{Remerciements} % (fold)
\label{cha:remerciements}


Ces trois années de thèse écoulées, je prends conscience de la chance que j'ai eu d’être entouré par chacun d’entre vous.


Je tiens tout d’abord à remercier mes directeurs de thèse Vincent Leclère et Frédéric Meunier.
Vous avez su me faire grandir personnellement et scientifiquement, m’aiguiller et m’offrir de l’autonomie sans cesser de veiller sur moi.
Surtout, je vous suis reconnaissant pour avoir toujours su me rassurer, calmer mes doutes et me permettre d'effectuer ma thèse dans les meilleures conditions.


J’adresse mes sincères remerciements à Jean-Philippe Gayon et Safia Kedad-Sidhoum pour avoir accepté de rapporter ma thèse et m’avoir montré comment améliorer ce manuscrit.
Je remercie également Nadia Brauner et Céline Gicquel d’être présentes pour ma soutenance.


Je souhaiterais adresser un remerciement particulier à Fabrice Bonneau, Directeur Général d’Argon Consulting pour l’intérêt des sujets proposés, sa disponibilité malgré son emploi du temps chargé et sa volonté d’investir et de s’investir dans cette recherche malgré les contraintes industrielles.
Comprendre les enjeux humains qui se cachaient dernière les mathématiques fut un réel plaisir.
Je remercie également toutes les personnes que j’ai pu côtoyer à Argon Consulting~: celles qui m’ont transmis leur passion et leur connaissance de la Supply Chain, les fonctions supports sans qui le quotidien ne serait pas si simple, ceux avec qui j’ai pu tisser des liens d’amitié et particulièrement les membres de l’OSS 117.


Je remercie également tous les chercheurs du CERMICS pour leur disponibilité et en particulier Jean-Philippe Chancelier sans qui l’informatique aurait été beaucoup plus difficile et Bernard Lapeyre et Julien Reygner pour leur aide sur les probabilités.
Je remercie également les doctorants qui ont su créer un environnement de travail et de détente en toute circonstance.
En particulier, je repense à nos conversations surréalistes à des heures tout aussi improbables, à ce séjour en Auvergne et cette \og{}petite ballade\fg{} qui aura finalement duré cinq heures et à notre séjour au ski.
Enfin, je souhaiterais adresser un remerciement tout particulier à Isabelle Simunic, la Secrétaire Générale du CERMICS.
Arrivé petit stagiaire en césure, tu m’as accueilli quelques semaines dans ton bureau, pendant lesquelles ta réserve de spéculoos a grandement diminuée. Tu t’es toujours pliée en quatre pour nous rendre la vie facile et tu nous montrais toujours ta bonne humeur malgré les difficultés de ton travail.


Je remercie également mes amis qui sont là aujourd’hui et qui me rendent la vie si agréable, que ce soit lors des sorties ou des soirées Donjons et Dragons.
Je me libère également aujourd’hui de toute blague sur mon statut étudiant.


Finalement, mes derniers remerciements vont à ma famille.
Merci Maman pour ton soutien, ta présence ton écoute.
Merci d’être aussi rock-and-roll.
Merci à mes frères et s\oe{}urs, Aleth, Louise, Simon et Matthias.
Merci d’être si déjantés et motivés pour tout (même le pire !) et de supporter ma maniaquerie. Merci à Natacha, la fiancée de Louise, pour nos parties de tennis hebdomadaires et pour être devenue ma troisième s\oe{}ur.
Enfin, tu n’es pas présent aujourd’hui malgré ton importance : merci Papa.
Tu as été là à toutes les étapes de ma vie et particulièrement pendant cette thèse.
Pendant deux ans et demi, j’ai pu avoir cette relation privilégiée avec toi lorsque je vivais sous ton toit.
Tu remplissais tous les rôles : le père, le colocataire, l’ami, le confident.
Si cette thèse a pu si bien se dérouler, je te le dois en grande partie.

