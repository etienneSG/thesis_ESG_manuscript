%!TEX root=../../thesis_ESG.tex
\chapter{Numerical experiments}
\label{chap:multi-sourcing:numerical-experiments}


We test our method on datasets from Argon Consulting’s clients.
Moreover, we compare the multi-sourcing currently in use by the client to the multi-sourcing computed from our method.
In addition to comparing their cost, several Key Performance Indicators (KPI) will be used as the  service level (fill rate and cycle), the proportion of multi-sourced items, etc.


\section{Simulation}


Simulation is based on real datasets from Argon Consulting’s clients.
We first retrieve the historical data and the current multi-sourcing to establish a comparison point.


First, we define plants data (capacities, internal production times, eligibility or not to be given the ability to produce items) and a forecast demand which is a function of the current period and of the past outcomes.
Then, we construct our model defined in \cref{chap:multi-sourcing:stochastic}.
Finally, we solve it and get a multi-sourcing decision.


The question of whether this solution is feasible or not (\ie satisfies th $\AVaR$ constraint) remains open, since, due to dataset sizes, the number of scenarios is small (100 in our experiments).
In order to show the validity of the obtained assignment, we fix it and compute Problem~\eqref{eq:stoch-multi-sourcing:linearized} with a larger number of scenarios (typically 10000).
This problem is a big linear program but has no integer variables.
If the solver succeeds in finding a feasible solution, it is likely that the $\AVaR$ constraint is satisfied and we then compute other KPI's like the fill rate or the cycle service level, or the multi-sourcing level.
If the solver proves the problem is infeasible, it is likely that the $\AVaR$ constraint is not satisfied and that the assignment is infeasible.
Finally, no conclusion can be drawn if time limit is reached before finding a solution.


\section{Instances}


\subsection{Datasets}


We use two instances provided by Argon Consulting's clients: Mel and Lux.


The Mel dataset is a toy problem.
It is a simplified dataset created by a client to enable the students from \'Ecole des Ponts to train and propose new approaches to multi-sourcing.


The Lux dataset comes from the luxury industry.
In the original dataset, there were 500 different items.
However, we keep only 100 items of Lux dataset, which represent 80\% of the production (and of the sales).
Due to limited computational resources, the client prefers that the 400 items excluded from Lux dataset be mono-sourced to concentrate our effort on potential multi-sourced items.


\cref{tab:multi-sourcing:instances-characteristics} summarizes the characteristics of the two instances used for tests.
\begin{table}[h]
\begin{tabular*}{\linewidth}{@{\extracolsep{\fill}}llr@{\extracolsep{\fill}}c@{\extracolsep{\fill}}rr@{\extracolsep{\fill}}c@{\extracolsep{\fill}}r@{\extracolsep{\fill}}}
\hline
\multicolumn{2}{c}{Instances characteristics} & \multicolumn{6}{c}{Instances}
\\
\cline{3-8}
&& \multicolumn{3}{c}{Mel} & \multicolumn{3}{c}{Lux}
\\
\hline
Number of plants & $\card{\plants}$ &&& $4$ &&& $27$
\\
Number of items & $\card{\REF}$ &&& $11$ &&& $100$
\\
Number of time steps & $\horizon$ &&& $3$ &&& $6$
\\
Historical demand & $\bar{\demand}_t^i$ & $4884$ &--& $19959$ & $0$ &--& $16996$
\\
Capacity of plants & $\capacity_p$ & $6156$ &--& $28969$ & $2213$ &--& $43416$ 
\\
Capacity for an item & $\ub_p^i$ & $1539$ &--& $7242$ & $721$ &--& $12753$ 
\\
Internal production time & $\rate_{pt}^i$ & $0.4$ &--& $1.1$ & $0.4$ &--& $9.0$
\\
Assignment cost & $\affect_p^i$ & $8392$ &--& $82196$ & $3516$ &--& $62234$ 
\\
\hline
\end{tabular*}
\caption{Instance characteristics}
\label{tab:multi-sourcing:instances-characteristics}
\end{table}


The parameter $\servicelvl$, which controls the service level is chosen in $\crbracket{70\%,85\%,95\%}$.
The number $m$ of scenarios used to solve (2SA-$m$) is fixed to $100$ due to tractability restrictions.
The time limit to return a solution has been set to $20$ minutes because the client wants to try several service levels.
Thus, the parameters for the heuristic defined in \cref{sec:multi-sourcing:stochastic:solving-method:heuristic} are $N=10$ and $\tau=120$ seconds.


\subsection{Demand distribution}
\label{sec:multi-sourcing:numerical-experiments:demand-distribution}


%\esgil{Completer après la réunion du 05/09 avec Fabrice, Frédéric et Vincent.}


As explained in \cref{chap:business-context}, we study the ability to face variations of the product mix and assume that we already know the cumulative working hours of production $\sum_{t=1}^{\horizon}\sum_{i\in\REF}\rate^i\va\demand_t^i$.


We generate randomness from historical data using ``expanded Dirichlet'' distributions as in \cref{chap:PDP:numerical-experiments}.
But this time, we assume that $(\rate^i\va\demand_t^i)$ follows an ``expanded Dirichlet'' distribution.
We refer to \cref{sec:PDP:numerical-experiments:drawing-realization-of-demand} for the properties of this distribution and to Algorithm~\ref{alg:demand:initial-sampling} to sample it.


The parameters of the ``expanded Dirichlet'' distribution used to generate the working hours of production are computed from the historical forecast demand and $\gamma$ is set equal to
\begin{equation}
  \gamma
  =
  v\frac
  {\sum_{t=1}^{\horizon}\sum_{i\in\REF} \sqrt{\frac{\bar{p}_0}{\bar{p}_t^i}-1}}
  {\sum_{t=1}^{\horizon}\sum_{i\in\REF} \bracket{\frac{\bar{p}_0}{\bar{p}_t^i}-1}}.
\end{equation}
The parameter $\gamma$ is defined from another parameter $v$ which we call \emph{volatility} and which represents the ratio between standard deviation and expectation of a random variable.
It can be easily interpreted.
Justification of such a choice for $\gamma$ can be found in \cref{sec:PDP:numerical-experiments:instances:fitting-parameters}.


As in \cref{chap:PDP:numerical-experiments}, results may depend on the number of generated scenarios.
If we had the time to implement method of \cref{sec:PDP:numerical-experiments:k-means}, we would like to apply the same scheme, that is:
\begin{itemize}
  \item use Algorithm~\ref{alg:demand:initial-sampling} to generate a sample of $M$ scenarios,
  \item initialize $K$ with $m=100$,
  \item use $K$-means algorithm to classify the $M$ scenarios into $m$ clusters.
\end{itemize}
Then, our set $\scenarios$ of representative scenarios used in \cref{eq:stoch-multi-sourcing:linearized} would be composed of the prototype of each cluster, with a probability equal to the number of scenarios in its cluster divided by $M$.


\section{Numerical results}
\label{sec:multi-sourcing:numerical-experiments:numerical-results}


% \esgil{Complete with KPI given by Fabrice and Pierre-Fabrice.}


C++11 has been chosen for the implementations and Gurobi 6.5.1~\citet{gurobi} was used to solve the model on a PC with 32 processor Intel\textregistered\ Xeon\texttrademark\ E5-2667 @ 3.30GHz and 192Go RAM.


There are two ways of measuring the KPI's.
First, they can be measured \emph{in-sample} (\ie with the $m=100$ scenarios used to decide multi-sourcing).
In this case, the obtained solution of Problem~\eqref{eq:stoch-multi-sourcing:linearized} is used to compute the KPI as if it stood for the whole possible realizations of the randomness.
On the other hand, they can be measured \emph{out-of-sample}.
We generate $m=10000$ new realizations of the demand.
Then, we solve Problem~\eqref{eq:stoch-multi-sourcing:linearized} with the assignment obtained from this optimization and these $m$ new realizations of the demand.


Simulation results for Mel dataset with a frontal resolution with a solver are given in \cref{tab:multi-sourcing:results:mel:without-heuristic} and with the use of the Algorithm~\ref{alg:multi-sourcing:MIP:heuristic} in \cref{tab:multi-sourcing:results:mel:with-heuristic}.
Simulation results for Lux dataset with the use of the Algorithm~\ref{alg:multi-sourcing:MIP:heuristic} are in \cref{tab:multi-sourcing:results:lux:with-heuristic}.


The inputs are decomposed as follows.
The parameter $\servicelvl$ is the value used to control the desired service level, \ie the risk level used in the $\AVaR$ constraint.
The volatility $v$ is the value used in the ``expanded Dirichlet'' distribution that generates the random realizations of the demand (see \cref{sec:multi-sourcing:numerical-experiments:demand-distribution}).


The outputs are decomposed as follows.
The row \emph{Solver Objective} (resp. \emph{Heuristic Objective}) gives the assignment costs returned by the solver (resp. the heuristic) at the end of the time allowed to find a solution.
This is the objective~\eqref{eq:stoch-multi-sourcing:linearized:objective} and it is expressed in k\euro{}.
The row \emph{Multi-sourcing} gives the percentage of multi-sourced families (\ie families produced on two plants or more) and the row \emph{Multi-sourcing (max)} gives the maximal number of plants producing one item.


The row $\prob\sqbracket{s\ge0}$ measures in-sample the probability (in \%) that inventory of a random item at a random time is positive.
\begin{equation}
  \prob\sqbracket{s\ge0}=\frac{1}{\horizon\times\card{\REF}\times\card{\scenarios}}\sum_{t=1}^{\horizon}\sum_{i\in\REF}\sum_{\omega\in\scenarios} \mathbb{1}_{\crbracket{\inventory_{t,\omega}^i\ge0}}
\end{equation}


The rows \emph{GAP}, \emph{Solver LB} \emph{Cont. relax.}, and \emph{Time for bound} are measured in-sample.
The row \emph{GAP} is the gap (in \%) between the objective value and the best known lower bound on the problem.
The row \emph{Solver LB} is the lower bound (in k\euro{}) found by the solver when the solver is frontally used to solve Problem~\eqref{eq:stoch-multi-sourcing:linearized}.
The row \emph{Cont. relax.} is the value (in k\euro{}) of the continuous relaxation of Problem~\eqref{eq:stoch-multi-sourcing:linearized}.
The row \emph{Time for bound} is the time (in seconds) needed to get the bound written on the previous row of the table.


The row \emph{Feasibility} says whether Problem~\eqref{eq:stoch-multi-sourcing:linearized} with the fixed obtained assignment and $m=10000$ scenarios is feasible (\emph{Yes}), infeasible (\emph{No}), or whether time limit was reached \emph{Limit}).


The rows \emph{Cycle service level} and \emph{Fill rate service level} are measured out-of-sample.
They are computed using \cref{eq:pdp:numerical-experiments:KPI:fill-rate-service-level} and \cref{eq:pdp:numerical-experiments:KPI:cycle-service-level} by replacing $w^i$ by
\begin{equation}\label{eq:multi-sourcing:numerical-experiments:KPI:weight}
  w^i=\frac{\rate^i}{\sum_{j\in\REF}\rate^j}.
\end{equation}
In this case, the item are weighted by their internal production time.
These two KPI's are averaged over the $m=10000$ scenarios and are expressed in \%.


The row \emph{Evaluation time} gives the time (in seconds) needed to solve Problem~\eqref{eq:stoch-multi-sourcing:linearized} with the fixed obtained assignment and $m=10000$ scenarios.
The time limit of the solver is set equal to 24 hours.


\begin{table}[h]
\begin{tabular*}{\linewidth}{@{\extracolsep{\fill}}c|r|r|r|r|r|r|l@{\extracolsep{\fill}}}
$\beta$ & \multicolumn{6}{c|}{Volatility $v$} & \multicolumn{1}{c}{Output}
\\
& \multicolumn{1}{c|}{1\%} & \multicolumn{1}{c|}{5\%} & \multicolumn{1}{c|}{10\%} & \multicolumn{1}{c|}{30\%} & \multicolumn{1}{c|}{50\%} & \multicolumn{1}{c|}{80\%} & 
\\ \hline
70\% & 953 & 967 & 1011 & 1034 & 1134 & 1381 & Solver Objective
\\
     & 45.5 & 54.5 & 54.5 & 63.6 & 100.0 & 72.7 & Multi-sourcing
\\
     & 2 & 2 & 3 & 3 & 4 & 3 & Multi-sourcing (max)
\\
     & 93.8 & 93.5 & 93.3 & 91.2 & 90.1 & 91.4 & $\prob\sqbracket{s\ge0}$ 
\\ \cdashline{2-8}[2pt/2pt]
     & Yes & Yes & Yes & Yes & No & Yes & Feasibility
\\
     & 71.2 & 74.4 & 73.4 & 78.2 & \multicolumn{1}{c|}{--} & 82.8 & Cycle service level
\\
     & 90.5 & 96.0 & 94.8 & 94.1 & \multicolumn{1}{c|}{--} & 92.4 & Fill rate service level
\\
     & 22 & 28 & 86 & 34 & 61 & 38 & Evaluation time
\\ \hline
85\% & 957 & 957 & 1000 & 1087 & 1188 & \multicolumn{1}{c|}{$\infty$} & Solver Objective
\\
     & 54.5 & 54.5 & 54.5 & 72.7 & 63.6 & \multicolumn{1}{c|}{--} & Multi-sourcing
\\
     & 3 & 2 & 2 & 3 & 3 & \multicolumn{1}{c|}{--} & Multi-sourcing (max)
\\
     & 98.5 & 98.1 & 97.6 & 96.5 & 95.9 & \multicolumn{1}{c|}{--} & $\prob\sqbracket{s\ge0}$
\\ \cdashline{2-8}[2pt/2pt]
     & 17.5 & 15.0 & 18.1 & 17.7 & 11.3 & \multicolumn{1}{c|}{--} & GAP
\\
     & 790 & 813 & 819 & 895 & 1054 & \multicolumn{1}{c|}{--} & Solver LB
\\
     & 1200 & 1200 & 1200 & 1200 & 1200 & \multicolumn{1}{c|}{--} & Time for bound
\\ \cdashline{2-8}[2pt/2pt]
     & Yes & Yes & Yes & Yes & Yes & \multicolumn{1}{c|}{--} & Feasibility
\\
     & 75.1 & 67.6 & 61.9 & 80.4 & \multicolumn{1}{c|}{--} & \multicolumn{1}{c|}{--} & Cycle service level
\\
     & 94.1 & 94.2 & 94.3 & 95.3 & \multicolumn{1}{c|}{--} & \multicolumn{1}{c|}{--} & Fill rate service level
\\
     & 17 & 21 & 40 & 25 & 43 & \multicolumn{1}{c|}{--} & Evaluation time
\\ \hline
95\% & 954 & 956 & 997 & 1173 & 1371 & \multicolumn{1}{c|}{$\infty$} & Solver Objective
\\
     & 54.5 & 54.5 & 72.7 & 72.7 & 81.8 & \multicolumn{1}{c|}{--} & Multi-sourcing
\\
     & 3 & 3 & 2 & 3 & 3 & \multicolumn{1}{c|}{--} & Multi-sourcing (max)
\\
     & 99.6 & 99.6 & 99.3 & 99.0 & 99.0 & \multicolumn{1}{c|}{--} & $\prob\sqbracket{s\ge0}$
\\ \cdashline{2-8}[2pt/2pt]
     & Yes & Yes & Yes & No & No & \multicolumn{1}{c|}{--} & Feasibility
\\
     & 72.1 & 76.6 & inf. & \multicolumn{1}{c|}{--} & \multicolumn{1}{c|}{--} & \multicolumn{1}{c|}{--} & Cycle service level
\\
     & 95.5 & 95.6 & inf. & \multicolumn{1}{c|}{--} & \multicolumn{1}{c|}{--} & \multicolumn{1}{c|}{--} & Fill rate service level
\\
     & 16 & 18 & 1204 & 863 & 1880 & \multicolumn{1}{c|}{--} & Evaluation time
\\ \hline
\end{tabular*}
\caption{Results for Mel dataset (frontally solved with the solver)}
\label{tab:multi-sourcing:results:mel:without-heuristic}
\end{table}


\begin{table}[h]
\begin{tabular*}{\linewidth}{@{\extracolsep{\fill}}c|r|r|r|r|r|r|l@{\extracolsep{\fill}}}
$\beta$ & \multicolumn{6}{c|}{Volatility $v$} & \multicolumn{1}{c}{Output}
\\
& \multicolumn{1}{c|}{1\%} & \multicolumn{1}{c|}{5\%} & \multicolumn{1}{c|}{10\%} & \multicolumn{1}{c|}{30\%} & \multicolumn{1}{c|}{50\%} & \multicolumn{1}{c|}{80\%} & 
\\ \hline
70\% & 1152 & 1176 & 1156 & 1136 & 1541 & 1640 & Heuristic Objective
\\
     & 90.9 & 100.0 & 100.0 & 100.0 & 100.0 & 100.0 & Multi-sourcing
\\
     & 4 & 3 & 4 & 4 & 4 & 4 & Multi-sourcing (max)
\\
     & 95.5 & 93.8 & 92.7 & 89.7 & 88.8 & 90.1 & $\prob\sqbracket{s\ge0}$
\\ \cdashline{2-8}[2pt/2pt]
     & Yes & Yes & Yes & Yes & Yes & Yes & Feasibility
\\
     & 83.6 & 83.4 & 72.1 & 82.7 & 88.7 & 85.3 & Cycle service level
\\
     & 96.5 & 97.5 & 96.2 & 95.9 & 96.6 & 94.0 & Fill rate service level
\\
     & 77 & 25 & 40 & 29 & 32 & 36 & Evaluation time
\\ \hline
85\% & 1152 & 1134 & 1146 & 1112 & 1230 & \multicolumn{1}{c|}{$\infty$} & Heuristic Objective
\\
     & 90.9 & 81.8 & 81.8 & 63.6 & 81.8 & 100.0 & Multi-sourcing
\\
     & 4 & 4 & 4 & 3 & 4 & 4 & Multi-sourcing (max)
\\
     & 97.7 & 97.2 & 96.6 & 96.2 & 95.7 & \multicolumn{1}{c|}{--} & $\prob\sqbracket{s\ge0}$
\\ \cdashline{2-8}[2pt/2pt]
     & 31.4 & 28.3 & 28.5 & 19.5 & 14.3 & \multicolumn{1}{c|}{--} & GAP
\\
     & 673 & 707 & 686 & 797 & 900 & 1144 & Cont. relax.
\\
     & 3.8 & 13.8 & 3.5 & 2.4 & 13.7 & 3.6 & Time for bound
\\ \cdashline{2-8}[2pt/2pt]
     & Yes & Yes & Yes & Yes & No & Yes & Feasibility
\\
     & 79.4 & 75.0 & 91.0 & 82.7 & \multicolumn{1}{c|}{--} & 86.5 & Cycle service level
\\
     & 95.2 & 96.3 & 98.4 & 95.7 & \multicolumn{1}{c|}{--} & 95.1 & Fill rate service level
\\
     & 42 & 35 & 33 & 27 & 55 & 27 & Evaluation time
\\ \hline
95\% & 1152 & 968 & 1021 & 1210 & 1396 & \multicolumn{1}{c|}{$\infty$} & Heuristic Objective
\\
     & 90.9 & 63.6 & 54.5 & 72.7 & 72.7 & 100.0 & Multi-sourcing
\\
     & 4 & 2 & 2 & 4 & 3 & 4 & Multi-sourcing (max)
\\
     & 99.2 & 99.8 & 99.8 & 99.0 & 99.1 & \multicolumn{1}{c|}{--} & $\prob\sqbracket{s\ge0}$
\\ \cdashline{2-8}[2pt/2pt]
     & Yes & Yes & Yes & No & No & No & Feasibility
\\
     & 76.2 & 76.7 & 72.7 & \multicolumn{1}{c|}{--} & \multicolumn{1}{c|}{--} & \multicolumn{1}{c|}{--} & Cycle service level
\\
     & 95.0 & 97.1 & 93.9 & \multicolumn{1}{c|}{--} & \multicolumn{1}{c|}{--} & \multicolumn{1}{c|}{--} & Fill rate service level
\\
     & 25 & 30 & 24 & 2333 & 2394 & 6739 & Evaluation time
\\ \hline
\end{tabular*}
\caption{Results for Mel dataset (solved with Algorithm~\ref{alg:multi-sourcing:MIP:heuristic})}
\label{tab:multi-sourcing:results:mel:with-heuristic}
\end{table}


\begin{table}[h]
\begin{tabular*}{\linewidth}{@{\extracolsep{\fill}}c|r|r|r|r|r|r|l@{\extracolsep{\fill}}}
$\beta$ & \multicolumn{6}{c|}{Volatility $v$} & \multicolumn{1}{c}{Output}
\\
& \multicolumn{1}{c|}{1\%} & \multicolumn{1}{c|}{5\%} & \multicolumn{1}{c|}{10\%} & \multicolumn{1}{c|}{30\%} & \multicolumn{1}{c|}{50\%} & \multicolumn{1}{c|}{80\%} & 
\\ \hline
70\% & 1392 & 1409 & 1420 & 1476 & 1624 & 1861 & Heuristic Objective
\\
     & 18.0 & 23.0 & 18.0 & 19.0 & 22.0 & 25.0 & Multi-sourcing
\\
     & 3 & 3 & 3 & 4 & 4 & 3 & Multi-sourcing (max)
\\
     & 99.4 & 99.3 & 99.2 & 98.7 & 98.6 & 98.5 & $\prob\sqbracket{s\ge0}$
\\ \cdashline{2-8}[2pt/2pt]
     & Yes & Yes & Yes & Yes & Yes & Yes & Feasibility
\\
     & 77.9 & 77.3 & 82.9 & 72.0 & 88.2 & 92.0 & Cycle service level
\\
     & 86.2 & 85.9 & 88.7 & 80.6 & 91.9 & 94.0 & Fill rate service level
\\
     & 4703 & 4293 & 3892 & 2212 & 2363 & 2154 &  Evaluation time
\\ \hline
85\% & 1392 & 1409 & 1407 & 1516 & 1708 & 1997 & Heuristic Objective
\\
     & 18.0 & 23.0 & 16.0 & 21.0 & 24.0 & 31.0 & Multi-sourcing
\\
     & 3 & 3 & 3 & 3 & 3 & 4 & Multi-sourcing (max)
\\
     & 99.9 & 99.8 & 99.8 & 99.6 & 99.5 & 99.4 & $\prob\sqbracket{s\ge0}$
\\ \cdashline{2-8}[2pt/2pt]
     & \multicolumn{1}{c|}{--} & 24.1 & 22.3 & 24.7 & \multicolumn{1}{c|}{--} & \multicolumn{1}{c|}{--} & GAP
\\
     & \multicolumn{1}{c|}{--} & 1070 & 1084 & 1142 & \multicolumn{1}{c|}{--} & \multicolumn{1}{c|}{--} & Cont. relax.
\\
     & 86400 & 28728 & 31277 & 51399 & 86400 & 86400 & Time for bound
\\ \cdashline{2-8}[2pt/2pt]
     & Yes & Limit & Yes & Yes & Yes & Yes & Feasibility
\\
     & 77.8 & \multicolumn{1}{c|}{--} & 82.0 & 85.2 & 90.5 & 92.7 & Cycle service level
\\
     & 86.1 & \multicolumn{1}{c|}{--} & 87.7 & 89.4 & 94.0 & 94.6 & Fill rate service level
\\
     & 4262 & 86402 & 5148 & 3692 & 3652 & 3044 & Evaluation time
\\ \hline
95\% & 1392 & 1420 & 1421 & 1624 & 1900 & 2382 & Heuristic Objective
\\
     & 18.0 & 18.0 & 19.0 & 22.0 & 26.0 & 34.0 & Multi-sourcing
\\
     & 3 & 3 & 3 & 4 & 4 & 5 & Multi-sourcing (max)
\\
     & 100.0 & 100.0 & 99.9 & 99.9 & 99.9 & 99.9 & $\prob\sqbracket{s\ge0}$
\\ \cdashline{2-8}[2pt/2pt]
     & Yes & Yes & Yes & Yes & Yes & Yes & Feasibility
\\
     & 77.8 & 80.3 & 81.1 & 86.4 & 90.6 & 92.7 & Cycle service level
\\
     & 85.9 & 87.2 & 87.1 & 90.1 & 93.1 & 94.5 & Fill rate service level
\\
     & 4025 & 5146 & 5209 & 5448 & 5716 & 5428 & Evaluation time
\\ \hline
\end{tabular*}
\caption{Results for Lux dataset (solved with Algorithm~\ref{alg:multi-sourcing:MIP:heuristic})}
\label{tab:multi-sourcing:results:lux:with-heuristic}
\end{table}


\section{Discussion}
\label{sec:multi-sourcing:results:discussion}


First, note that we do not provide results for Lux dataset without the heuristic.
Indeed, we were not able to find a solution in the allowed time (20 minutes).
Moreover, for the Lux dataset, it is even impossible to compute the continuous relaxation in less than 24 hours for a volatility equal to $1\%$, $50\%$ and $80\%$.


Second important point, on small datasets, the frontal use of a solver always gives better results than Algorithm~\ref{alg:multi-sourcing:MIP:heuristic} even if this gap decreases with the increase of volatility.
However, on huge datasets, the solver was never able to get a feasible solution.


Although it is not a proof, the values returned by $\prob\sqbracket{s\ge0}$ show on some examples that the satisfaction of the $\AVaR$ constraint implies the satisfaction of the probabilistic constraint.


As shown by the computational times, evaluating the quality solution with fixed assignments is hard.
It may take up to two hours while the problem has no integer variables.


On Mel instances, we see that solving on a small set of scenarios is too optimistic when computing the multi-sourcing.
Indeed, several assignments prove to be infeasible of the larger set of scenarios.


Regarding results on Lux instances, we see that Algorithm~\ref{alg:multi-sourcing:MIP:heuristic} is not that bad.
When we succeed in finding the continuous relaxation of Problem~\eqref{eq:stoch-multi-sourcing:linearized} with $m=100$ scenarios, the GAP to optimality is at most 25\%.
For small values of the volatility $v$, playing with the value of $\servicelvl$ has little impact on the objective value and on service level.
For greater values of the volatility, playing with the values of $\servicelvl$ enables us to reach the desired service levels.
However, when $\servicelvl$ becomes close to $100\%$, our method becomes very sensitive to the sampling and approaches like robust ones may be more appropriate.



% \esgil{Refaire des tests avec des valeur plus faibles de $\servicelvl$}


\section{Comparison with the current multi-sourcing of Argon Consulting's client}


Regarding the whole items list (\ie the 500 families), Argon Consulting's client has almost 25\% of multi-sourced items.
In the 100 considered items, it is about 80\% of the items.
With our approach, we show that we can drastically reduce the number of multi-sourced items (and thus the multi-sourcing costs) while keeping a high service level.
As mentioned in \cref{sec:multi-sourcing:results:discussion}, some complementary implementations can be made in order to improve the method and ensure the validity of the approach.


