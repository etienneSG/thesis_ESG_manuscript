%!TEX root=../../thesis_ESG.tex
\chapter{Deterministic multi-sourcing}
\label{chap:multi-sourcing:deterministic}

\section{Introduction}

% In \cref{part:production planning}, we studied production planning problem which are mid-term decisions.
% A main constraint of this problem was the flexibility of the assembly line (which can be easily understand as the flexibility of any production resource).
% Thus, it was an input of the previous problem.
% This part aims at deciding the flexibility of plants (or any other production resources) and is a long-term decision.

\vlil{Deciding which plants should have the ability to produce a (new) item is a long-term decision that companies have
to answer when expanding their product portfolio. Note that several plants may be able to produce the same item that is why this problem is called \emph{multi-sourcing}.}
Deciding production multi-sourcing is a basic decision when companies expand their product portfolio.
Production multi-sourcing is a long-term decision consisting in deciding if a plant should have the ability to produce an item.
Note that several plants may be able to produce the same item that is why this problem is called \emph{multi-sourcing}.
Because of their costs and of the time needed to implement them, multi-sourcing decisions cannot be easily changed.
Thus, they have a high impact on competitiveness when deciding production planning (mid-term decision) and scheduling (short-term decision).
One could think that a complete flexibility of each plant is the safest solution to ensure future high service level.
However, multi-sourcing the production of every item leads to unnecessary high affectation costs.
The challenge is to find a trade-off between affectation cost and demand satisfaction.


Studying this problem, Argon Consulting aims at getting a versatile model which can be used in many situation.
Thus, the model must not be too dependent of the industry.
For this reason, another objective is to address a model which captures the essential parts and difficulties of the multi-sourcing problem
and which can be easily adapted to particular clients cases.
For example, inventory cost at the end end of a period will not be taken into account but a particular client may want to bound it.
Thus the model have to be able to add this constraint.
Resolution method must also be able to adapt to model modifications.


\medskip


We considers a set $\plants$ of plants producing a set $\REF$ of items over $\horizon$ periods.
There is an upper bound $\capacity_{pt}$ on the total period production of plant $p$ (summed over all items).

Giving a plant $p$ the ability to produce an item $i$ has a cost $\affect_p^i$.
This cost is paid once and for all for the whole horizon.
When a plant $p$ is able to produce item $i$, there is an upper (resp. lower) bound $\ub_{pt}^i\ge 0$ (resp. $\lb_{pt}^i\ge 0$) on the production of item $i$ in plant $p$ at period $t$.
The capacity needed (in time units) to produce one unit of item $i$ in plant $p$ in period $t$ is $\rate_{pt}^i>0$.


The production of item $i$ (summed over all plants) must satisfy a demand $\demand_t^i$ at the end of period $t$.
When production of an item $i$ is not used to satisfy the demand, it can be stored and incurs no cost.
For each item $i$, there is an initial inventory $\inventory_0^i\in\RR_+$.


The goal is to satisfy the whole demand at minimum cost.


We call this problem the \emph{deterministic multi-sourcing problem}.


\medskip


Some parameters or their time dependence may seem unusual.
First, the lower bound on the production.
When a company give a plant $p$ the ability to produce an item $i$, some of them impose a minimal production $\lb_{pt}^i$ for period $t$.
In this particular case, these minimal productions are usually constant and positive during the first periods after affectation and null during the last periods.
Second, the capacity needed to produce one item may depends on time.
This is often the case when people are involved in production since they become more efficient as time goes by.
Finally, as we are interested in multi-sourcing decisions, we do not consider production nor holding costs.

Thus, comparing future holding costs to actual affectation cost is hard.
However, extending the model by adding holding or production cost or bounding the inventory between each period is straightforward.


\medskip


We also define the \emph{uniform deterministic multi-sourcing problem}. Contrary to \cref{chap:PDP:deterministic}, this model does not rely on applications on real dataset but on technical considerations for proofs. In this problem, we set
$$
\rate_{pt}^i = 1
,\quad
\lb_{pt}^i = 0
,\quad
\ub_{pt}^i = \capacity_{pt} = 1
.
$$

\vlil{Je pense qu'il faudrait annoncer les résultats principaux du chapitre}


\section{Bibliography}


\esgil{TODO}

Johnzen et al (2010): mesure de flexibilité
 
$f=\frac{\sum_{i\in\REF}\sum_{p\in\plants}\open_p^i}{\card{\REF\times\plants}}$ (Et défini en pondérant par la production de chaque item)


\section{Model formulations}


In this section, we introduce a mix integer program which models the deterministic multi-sourcing problem.
We introduce the following decisions variables.
The quantity of item $i$ produced at period $t$ by plant $p$ is denoted by $\quantity_{pt}^i$ and the inventory at the end of the period is denoted by $\inventory_t^i$ .
We also introduce a binary variable $\open_p^t$ which takes the value 1 if plant $p$ is given the ability to produce item $i$.


The deterministic multi-sourcing problem can be written as
\begin{subequations}\label{eq:det-multi-sourcing}
  \begin{align+}
    \min\quad & \rlap{$\ds \sum_{i\in\REF}\sum_{p\in\plants}\affect_p^i \open_p^i$}
    \label{eq:det-multi-sourcing:objective}
    \\
    \st\quad & \ds \inventory_t^i = \inventory_{t-1}^i + \sum_{p\in\plants}\quantity_{pt}^i - \demand_t^i && \forall t\in\range{\horizon},\ \forall i\in\REF,
    \label{eq:det-multi-sourcing:inventory-dynamic}
    \\
    & \ds \sum_{i\in\REF}\rate_{pt}^i\quantity_{pt}^i \leq \capacity_{pt} && \forall t\in\range{\horizon},\ \forall p\in\plants,
    \label{eq:det-multi-sourcing:capacity}
    \\
    & \ds \lb_{pt}^i \open_p^i \le \rate_{pt}^i\quantity_{pt}^i \le \ub_{pt}^i \open_p^i && \forall t\in\range{\horizon},\ \forall p\in\plants, \forall i\in\REF,
    \label{eq:det-multi-sourcing:big-M}
    \\
    & \ds \inventory_t^i \ge 0 && \forall t\in\range{\horizon},\ \forall p\in\plants, \forall i\in\REF.
    \label{eq:det-multi-sourcing:positivity}
    \\
    & \ds \open_p^i \in \crbracket{0,1} && \forall p\in\plants, \forall i\in\REF,
    \label{eq:det-multi-sourcing:boolean}
  \end{align+}
\end{subequations}


Objective~\eqref{eq:det-multi-sourcing:objective} minimizes the affectation costs.
Constraint~\eqref{eq:det-multi-sourcing:inventory-dynamic} is the inventory balance.
Capacity of each plant is ensured by constraint~\eqref{eq:det-multi-sourcing:capacity}.
Constraint~\eqref{eq:det-multi-sourcing:big-M} is both a ``big-M'' constraint and a bound on the production of each item in each plant.
Note that without loss of generality, we can suppose that $\ub_{pt}^i \le \capacity_{pt}$ for each period $t$, each plant $p$ and each item $i$.


\medskip


The simplifications made for the uniform deterministic multi-sourcing problem lead to the following linear program
\begin{subequations}\label{eq:det-multi-sourcing}
  \begin{align+}
    \min\quad & \rlap{$\ds \sum_{i\in\REF}\sum_{p\in\plants}\affect_p^i \open_p^i$}
    \label{eq:det-multi-sourcing:objective}
    \\
    \st\quad & \ds \inventory_t^i = \inventory_{t-1}^i + \sum_{p\in\plants}\quantity_{pt}^i - \demand_t^i && \forall t\in\range{\horizon},\ \forall i\in\REF,
    \label{eq:det-multi-sourcing:inventory-dynamic}
    \\
    & \ds \sum_{i\in\REF}\quantity_{pt}^i \leq 1 && \forall t\in\range{\horizon},\ \forall p\in\plants,
    \label{eq:det-multi-sourcing:capacity}
    \\
    & \ds \quantity_{pt}^i \le \open_p^i && \forall t\in\range{\horizon},\ \forall p\in\plants, \forall i\in\REF,
    \label{eq:det-multi-sourcing:big-M}
    \\
    & \ds \open_p^i \in \crbracket{0,1} && \forall p\in\plants, \forall i\in\REF,
    \label{eq:det-multi-sourcing:boolean}
    \\
    & \ds \quantity_{pt}^i,\ \inventory_t^i \ge 0 && \forall t\in\range{\horizon},\ \forall p\in\plants, \forall i\in\REF.
    \label{eq:det-multi-sourcing:positivity}
  \end{align+}
\end{subequations}



\section{NP-completeness}
\label{sec:multi-sourcing:deterministic:NP-completeness}


The deterministic multi-sourcing problem is $\NP$-hard in the strong sense.


\begin{thm}\label{thm:deterministic-multi-sourcing:strong-NP-hard}
Deciding if there is a solution with cost $\card{\REF}$ of the uniform deterministic multi-sourcing problem is $\NP$-complete even with unit affectation costs and one period.
\end{thm}


Reducing 3-partition problem to uniform deterministic multi-sourcing problem, we show that deciding if there is a solution with cost $\card{\REF}$ of the deterministic multi-sourcing problem is $\NP$-complete. We remind that the 3-partition problem consists in deciding whether a given multiset $\crbracket{a_1,\ldots,a_{3m}}$ of integers can be partitioned into triples that all have the same sum. This problem is known to be $\NP$-complete in the strong sense (see~\cite{Garey1979}) even if $\frac{B}{4} < a_i < \frac{B}{2}$ for each $i$ with $B=\frac{1}{m}\sum_{i=1}^{3m}a_i$.



\begin{proof}
Let $\crbracket{a_1,\ldots,a_{3m}}$ be an instance of the 3-partition problem such that $\frac{B}{4} < a_i < \frac{B}{2}$ for each $i$ with $B=\frac{1}{m}\sum_{i=1}^{3m}a_i$.
We reduce polynomially this problem to an instance of the deterministic multi-sourcing problem.
%Without loss of generality, we can assume that sum of the $a_i$'s is positive.
We set
$$
  \horizon=1
  ,\quad
  \plants=\range[1]{m}
  ,\quad
  \REF=\range[1]{3m},
$$
$$
  \affect_p^i=1
  ,\quad
  \demand_1^i=a_i
  ,\quad
  \rate_{p,1}^i=1
  ,\quad
  \lb_{p,1}^i=0
  ,\quad
  \ub_{p,1}^i=\capacity_{p,1}=B.
$$
Thus, if we have a solution for the 3-partition problem, finding a solution with cost $\card{\REF}$ of the deterministic multi-sourcing problem is straightforward.
Conversely, if we have a solution with cost $\card{\REF}$ of the deterministic multi-sourcing problem with these parameters, $\frac{\capacity_{p,1}}{4}=\frac{B}{4} < a_i$ ensures that there are at most three items per plant.
Since each item is affected to at least one plant, we got a collection of $m$ triples.
Plants having the same capacity and sum of plant capacities being equal to sum of demands, each triple has the same sum.
Thus, we get a solution of the 3-partition problem.
The conclusion follows from the fact that the 3-partition problem is $\NP$-complete in the strong sense even if $\frac{B}{4} < a_i < \frac{B}{2}$ for each $i$.
\end{proof}


\cref{thm:deterministic-multi-sourcing:strong-NP-hard} prove the strong $\NP$-hardness but the proof require instances of the deterministic multi-sourcing problem with many plants.
One can ask for simpler cases.
\cref{thm:deterministic-multi-sourcing:NP-hard:2-plants} shows that deterministic multi-sourcing problem remains $\NP$-hard with one period and two plants and \cref{prop:deterministic-multi-sourcing:polynomial-cases} gives some polynomial cases.


\begin{thm}\label{thm:deterministic-multi-sourcing:NP-hard:2-plants}
Deciding if there is a solution with cost $\card{\REF}$ of the uniform deterministic multi-sourcing problem is $\NP$-complete even with unit affectation costs, one period and two plants.
\end{thm}


Reducing partition problem to the uniform deterministic multi-sourcing problem, we show that deciding if there is a solution of the uniform deterministic multi-sourcing problem is $\NP$-complete. We remind that the partition problem is the task of deciding whether a given set of positive integers can be partitioned into two subsets that have the same sum. This problem is known to be $\NP$-complete (see~\cite{Garey1979}).


\begin{proof}
Let $\crbracket{a_1,\ldots,a_m}$ be an instance of the partition problem.
We reduce polynomially this problem to an instance of the deterministic multi-sourcing problem.
We set
$$
  \horizon=1
  ,\quad
  \plants=\crbracket{1,2}
  ,\quad
  \REF=\range[1]{m},
$$
$$
  \affect_p^i=1
  ,\quad
  \demand_1^i=a_i
  ,\quad
  \rate_{pt}^i=1
  ,\quad
  \lb_{pt}^i=0
  ,\quad
  \ub_{pt}^i=\capacity_{pt}=\frac{1}{2}\sum_{i=1}^{m}a_i.
$$
Thus, we have a solution with cost $\card{\REF}$ for the partition problem if and only if there is a solution to the deterministic multi-sourcing problem with these parameters.
Indeed, each $a_i$ being positive, each item is affected to at least one plant
Thus, if we have a solution for the partition problem, finding a solution with cost $\card{\REF}$ of the deterministic multi-sourcing problem is straightforward.
Conversely, if we have a solution with cost $\card{\REF}$ of the deterministic multi-sourcing problem with these parameters, positivity of the $a_i$ ensures that each item is affected to at least one plant.
Cost of the solution being equal to $\card{\REF}$, each item is affected to exactly one plant.
Plants having the same capacity and sum of plant capacities being equal to sum of demands, each subset define by the affectation has the same sum.
Thus, we get a solution of the partition problem.
The conclusion follows from the fact that the partition problem is $\NP$-complete.
\end{proof}


\begin{prop}\label{prop:deterministic-multi-sourcing:polynomial-cases}
The following special cases of the deterministic multi-sourcing problem are polynomial:
\begin{enumerate}
  \item deterministic multi-sourcing problem with a single plant ($\plants=\crbracket{1}$),
  \item deterministic multi-sourcing problem without affectation cost ($\affect_p^i=0$),
  \item deterministic multi-sourcing problem with infinite capacities ($\ub_{pt}^i=\capacity_{pt}=+\infty$).
\end{enumerate}
\end{prop}


\begin{proof}
\emph{Case 1: deterministic multi-sourcing problem with a single plant.}

For each item $i$, we set
$$
\open_1^i=
\left\{
\begin{array}{l}
1\mbox{ if there exists }t\in\range{\horizon}\mbox{ such that }\demand_t^i>0,\\
0\mbox{ otherwise}.
\end{array}
\right.
$$
Then, we solve the resulting linear program to get the optimal solution of the deterministic multi-sourcing problem.
(It eventually\vl{may} returns that problem is infeasible.)

\medskip

\emph{Case 2: deterministic multi-sourcing problem without affectation cost.}

For each plant $p$ and each item $i$, we set $\open_p^i=1$.
These decisions do not affect the cost in the objective function.
Then, we solve the resulting linear program to get the optimal solution of the deterministic multi-sourcing problem.
(It may returns that problem is infeasible.)

\medskip

\emph{Case 3: deterministic multi-sourcing problem with infinite capacity.}

For each item $i$ we choose a plant $p(i)$ among $\argmin_{p\in\plants}\bracket{\affect_p^i}$.
Then, we set
$$
\open_p^i=
\left\{
\begin{array}{l}
1\mbox{ if }p=p(i),\\
0\mbox{ otherwise}.
\end{array}
\right.
$$
Then, we solve the resulting linear program to get the optimal solution of the deterministic multi-sourcing problem.
(It may returns that problem is infeasible.)
\end{proof}



