\chapter{Stochastic multi-sourcing}
\label{chap:multi-sourcing:stochastic}


\section{Motivations}
\label{sec:multi-sourcing:stochastic:motivations}


In \cref{chap:multi-sourcing:deterministic}, data are deterministic.
In practice, part of data is uncertain.
Indeed, multi-sourcing decisions are long-term decision and it is unlikely that forecast demand is perfectly accurate.
As in production planning decision, uncertainty may come from efficiency of machine (and affect time needed for the production of one item), capacities (due to breakdown, strike) or any other part of the supply chain like the procurement of raw materials.
When making multi-sourcing decisions, Argon Consulting identifies that uncertainty on forecast demand is the main issue for its clients.
More precisely, since it addresses the question of flexibility (see \cref{chap:business-context}), the relevant uncertainty comes from variation in the product mix and not from variations of global volume of demand nor from other sources.


Dealing with uncertainty leads to consider possible undelivered quantities.
It goes from the more conservative which does not allow stock-out to method controlling stock-out probability, stock-out quantity or other relevant indicators.
Discussions with Argon Consulting raise two characteristics of undelivered quantities.
First, multi-sourcing decisions should lead to ``high long term service level''.
Indeed, multi-sourcing decisions take time to be implemented.
Thus, service level on the first periods can be low due to augmentation of capabilities and should be less relevant than service level in steady state.
The measure used for service level should not be too sensitive to sparse failure at the beginning.
Second, worst cases be not ``not too bad''.
Multi-sourcing decisions are long-term decisions and cannot be easily changed.
However, when bad outcomes occur, it should be controlled and remain reasonable since it will impact the company competitiveness for long periods.
Thus, when making service level decisions, optimization should take into account worst cases and not just optimize good ones.


These two characteristics lead to ill-defined constraints.
Several models are possible and will be discuss in \cref{sec:multi-sourcing:stochastic:model:discussion}.


\medskip


As in its deterministic version, the problem considers a set $\plants$ of plants producing a set $\REF$ of items over $\horizon$ periods.
There is an upper bound $\capacity_{pt}$ on the total period production of plant $p$ (summed over all items).

Giving a plant $p$ the ability to produce an item $i$ has a cost $\affect_p^i$.
This cost is paid once and for all for the whole horizon.
When a plant $p$ is able to produce item $i$, there is an upper (resp. lower) bound $\ub_{pt}^i\ge 0$ (resp. $\lb_{pt}^i\ge 0$) on the production of item $i$ in plant $p$ at period $t$.
The capacity needed (in time units) to produce one unit of item $i$ in plant $p$ in period $t$ is $\rate_{pt}^i>0$.


The production of item $i$ (summed over all plants) must satisfy a random demand $\va\demand_t^i$ at the end of period $t$.
When production of an item $i$ is not used to satisfy the demand, it can be stored and incurs no cost.
For each item $i$, there is an initial inventory $\inventory_0^i\in\RR_+$.


We introduce a parameter $\servicelvl\in\sqbracket{0,1}$ which controls the desired service level.
As service level measure is not already defined, this parameter is not the service level but its variations should imply same sense variations of the service level.


The goal is to satisfy the desired service level at minimum cost.


We called this problem the \emph{stochastic multi-sourcing problem}.


\section{Bibliography}


\section{Model formulation}


Since some constraints of the problem are ill-defined, we first introduce preliminary definitions, then we present the model and finally discuss the choices made.


\subsection{Preliminary definitions}


In order to measure and control the service level, we use the \emph{Average Value at Risk} which is a \emph{coherent measure of risk} widely studied by Rockafellar and Uryasev (see for example \cite{Rockafellar2000,Rockafellar2002}) and chose the definition given in \cite{Follmer2004}.
In financial context, measures of risk are used to quantify the risk of a position.


\begin{defn}
Fix some level $\lambda\in\left]0,1\right[$. For a random variable $\va X:\scenarios\to\RR$, we define its \emph{Value at Risk at level $\lambda\in\bracket{0,1}$} as
$$\VaR_{\lambda}\bracket{\va X} = \inf\crbracket{m\in\RR\left|\prob\crbracket{\va X+m<0}\le\lambda\right.}.$$
\end{defn}
For the inventory, $\VaR_{1-\servicelvl}\bracket{\va\inventory}$ might correspond to the safety stock in the case of a cycle service level $\servicelvl$.
More precisely, adding this quantity to the inventory keeps the probability of a stock-out below the level $1-\servicelvl$.
The main drawback of the Value at Risk is that it only controls the probability of a stock-out and not its size.
Thus, we use the following coherent measure of risk defined from the Value at Risk.


\begin{defn}
Fix some level $\lambda\in\left]0,1\right[$. For a random variable $\va X:\scenarios\to\RR$, we define its \emph{Average Value at Risk at level $\lambda\in\bracket{0,1}$} as
$$\AVaR_{\lambda}\bracket{\va X} = \frac{1}{\lambda}\int_0^{\lambda}\VaR_{\gamma}\bracket{\va X}\diff\gamma.$$
\end{defn}
For the inventory, $\AVaR_{1-\servicelvl}\bracket{\va\inventory}$ seems like the safety stock in the case of a fill rate service level $\servicelvl$.
More precisely, $\AVaR_{1-\servicelvl}\bracket{\va\inventory}\le 0$ means the average of safety stocks needed to ensure service level from $\servicelvl$ to 1 is negative (\ie it does not need safety stocks).


\begin{figure}[h]
  \centering
  \includegraphics{main/multi-sourcing/images/avar-example.tikz}
  \caption{Example of risk measures for the inventory}
  \label{fig:avar-examples}
\end{figure}


\cref{fig:avar-examples} represents the Value at Risk and the Average Value at Risk in the case of a Gaussian distribution of the stock $\va\inventory_t^i$


\subsection{Model}


In order to solve the stochastic multi-sourcing problem, we introduce the following decision variables.
The quantity of item $i$ produced at period $t$ by plant $p$ is denoted by $\va\quantity_{pt}^i$ and the inventory at the end of the period is denoted by $\va\inventory_t^i$ .
We also introduce a binary variable $\open_p^t$ which takes the value 1 if plant $p$ is given the ability to produce item $i$.
All these variables are random and may depend on the past realizations of the random demand $\bracket{\va\demand_1^j,\ldots,\va\demand_{t-1}^j}_{j\in\REF}$.


We model the stochastic multi-sourcing problem as
\begin{subequations}\label{eq:stoch-multi-sourcing}
  \begin{align+}
    \min\quad & \rlap{$\ds \sum_{i\in\REF}\sum_{p\in\plants}\affect_p^i \open_p^i$}
    \label{eq:stoch-multi-sourcing:objective}
    \\
    \st\quad & \ds \va\inventory_t^i = \va\inventory_{t-1}^i + \sum_{p\in\plants}\va\quantity_{pt}^i - \va\demand_t^i && \forall t\in\range{\horizon},\ \forall i\in\REF,
    \label{eq:stoch-multi-sourcing:inventory-dynamic}
    \\
    & \ds \sum_{i\in\REF}\rate_{pt}^i\va\quantity_{pt}^i \leq \capacity_{pt} && \forall t\in\range{\horizon},\ \forall p\in\plants,
    \label{eq:stoch-multi-sourcing:capacity}
    \\
    & \ds \lb_{pt}^i \open_p^i \le \rate_{pt}^i\va\quantity_{pt}^i \le \ub_{pt}^i \open_p^i && \forall t\in\range{\horizon},\ \forall p\in\plants, \forall i\in\REF,
    \label{eq:stoch-multi-sourcing:big-M}
    \\
    & \AVaR_{1-\servicelvl}\bracket{\va\inventory_t^i} \le 0 && \forall t\in\range{\horizon},\ \forall i\in\REF,
    \label{eq:stoch-multi-sourcing:AVaR}
    \\
    & \ds \open_p^i \in \crbracket{0,1} && \forall p\in\plants, \forall i\in\REF,
    \label{eq:stoch-multi-sourcing:boolean}
    \\
    & \ds  \va\quantity_{pt}^i\ \mbox{is}\ \Sfield{\bracket{\va\demand_1^i,\ldots,\va\demand_{t-1}^i}_{i\in\REF}}\mbox{--measurable} && \forall t\in\range{\horizon},\ \forall i\in\REF.
    \label{eq:stoch-multi-sourcing:measurability}
  \end{align+}
\end{subequations}


Objective~\eqref{eq:stoch-multi-sourcing:objective} still minimizes the affectation costs.
Constraints~\eqref{eq:stoch-multi-sourcing:inventory-dynamic}, \eqref{eq:stoch-multi-sourcing:capacity}, \eqref{eq:stoch-multi-sourcing:big-M} have the same meaning than their deterministic counterpart.
Constraint~\eqref{eq:stoch-multi-sourcing:AVaR} enables to control the service level using Average Value at Risk.
It will be discussed in \cref{sec:multi-sourcing:stochastic:model:discussion}.
Last constraint~\eqref{eq:stoch-multi-sourcing:measurability} of the program, written as a {\em measurability constraint}, means that the values of the variables $\va\quantity_{pt}^i$ can only depend on the values taken by the demand before time $t$ (the decision maker does not know the future).
Every constraint of the problem, except the Average Value at Risk constraint~\eqref{eq:stoch-multi-sourcing:AVaR}, hold almost surely.


\subsection{Expectation, robust, probabilistic and average value at risk constraints}
\label{sec:multi-sourcing:stochastic:model:discussion}


As explained in \cref{sec:multi-sourcing:stochastic:motivations}, multi-sourcing decision must lead to ``high long term service level'' and a good control on ``worst cases''.
These constrains are ill-defined.
Many way of model it are possible.
We choose to use the Average Value at Risk and compare its pros and cons to the ones of expectation, robust and probabilistic constraints.
\cref{tab:constraint-properties-comparison} summarizes this comparison.


\begin{table}[h]
  \centering
  \begin{tabular*}{\linewidth}{@{\extracolsep{\fill}}lcccc@{\extracolsep{\fill}}}
  \hline
  \multicolumn{1}{c}{Properties} & \multicolumn{4}{c}{Constraints} \\
  \cline{2-5}
                                         & $\AVaR$      & Expectation  & Robust       & Probabilistic \\
  \hline
  Easy writing                           & \bulletminus & \bulletplus  & \bulletminus & \bulletminus \\ 
  Easy linearization                     & \bulletplus  & \bulletplus  & \bulletplus  & \bulletminus \\
  Easy interpretation                    & $\sim$       & \bulletminus & \bulletplus  & \bulletplus  \\
  Control proportion of worst cases      & \bulletplus  & \bulletminus & \bulletminus & \bulletplus  \\
  Control value of worst cases           & \bulletplus  & \bulletminus & \bulletplus  & \bulletminus \\
  Control increase of objective function & \bulletplus  & \bulletminus & \bulletminus & \bulletplus  \\ 
  Few feasibility issues                 & $\sim$       & \bulletplus  & \bulletminus & $\sim$       \\ 
  Computational complexity               & $\sim$       & \bulletplus  & \bulletplus  & \bulletminus \\
  \hline
  \end{tabular*}
  \caption{Comparison of constraint properties}
  \label{tab:constraint-properties-comparison}
\end{table}


We do not pretend to state general truth but rather general trend on common cases.
We propose eight criterion to compare the constraints.
First, the ease to write the constraint and to linearize it if needed.
In many cases, expectation constraint is very easy to write since it is often a sum, an integral or a combination of the two.
More, it is often already linear making it easy to use.
On the other hand, probabilistic constraint is hard to write and to linearize since it is hardly never convex.
Robust and $\AVaR$ constraint are both not easy to write.
Every possible outcome must be considered in robust case and definition of $\AVaR$ makes it hard to use.
However, robust and $\AVaR$ constraints are convex.
Using scenarios often enables to get linear formulation of robust constraint of reasonable size and as shown in \cref{sec:multi-sourcing:stochastic:avar-linearization}, $\AVaR$ can also be linearized.


Interpretation of robust and probabilistic constraints are quite easy.
Even though expectation constraint seems easy to interpret, in practice, knowing the behavior of the mean does not give any information on the behavior of the other outcome.
Thus, it is very hard to say something from only this information.
Understanding $\AVaR$ constraint at level $\lambda\%$ is not straightforward.
However, it means that $\bracket{1-\lambda}$ per cent of cases satisfies the constraint and that the mean over the $\lambda$ remaining per cent also satisfy the constraint.


Thus, we easily see that $\AVaR$ constraint enables to control the proportion of bad cases and their value.
On the other extreme, expectation constraint does not control any of them.
Being very conservative, robust constraint does not enable to control the proportion of bad case: you must cover every outcome but it ensure that every outcome value satisfies the constraint.
On the other hand, probabilistic constraint enable to choose the proportion of worst case but it is done at costs of control on their value.


Another advantage of $\AVaR$ and probabilistic constraint is the possibility to choose the level of conservativeness.
Indeed, the decision-maker can easily see the price implied by the service level constraint and eventually relax partially this constraint with a single parameter.
On the other hand, there is not any way of changing the robust and probabilistic constraint behavior.


Feasibility issues of each of this constraint is directly linked to it level of conservativeness.
Robust constraint is the most conservative and a single very bad outcome lead to an infeasible model.
At the other extreme, expectation constraint is much more easy to satisfy.
Choice of the type of constraint for this criteria is strongly linked to the distribution of the random variable.


Finally, as expected, the flexibility offered by $\AVaR$ and probabilistic constraint have a price paid in computational complexity.
On the other hand, expectation and robust constraint are often easier to deal with.


\medskip


To summarize, we choose $\AVaR$ constraint because of its linearization properties and the possible interpretation for industrial application.
Moreover, as wanted by Argon Consulting, $\AVaR$ enables to control proportion of worst cases and their value.
The price for this is the computational complexity.


\section{Simple example}


\section{Theoretical results}

\subsection{Linearization of $\AVaR$}
\label{sec:multi-sourcing:stochastic:avar-linearization}

\subsection{Bender decomposition}

\esgil{To do}


