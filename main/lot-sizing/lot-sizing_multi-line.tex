\chapter{Production on several lines}
\label{chap:lot-size:several-lines}



\cref{chap:lot-size:single-line} was dedicated to decide the optimal cycle stocks when only a single is involved.
In this chapter, we decide the optimal cycle stocks when several lines are involved in production.



\section{Motivations}
\label{sec:lot-size:several-lines:motivations}


Context is almost the same than in \cref{chap:lot-size:single-line}.
A company aims at reducing it cycle stocks (which globally helps to decrease the inventory) subject to flexibility constraint of the assemblies lines.
In many applications, a company has many assembly lines and production of a single item may be done on several lines.
This kind of configuration is very common and are the consequences of multi-sourcing decisions previously made (see \cref{sec:business-context:argon:multi-sourcing} for a description of multi-sourcing issues).
These lines might be is the same plants as well in different ones.
But, in both cases, it enables to increase flexibility of production and the workload between lines.


As in single line case, assembly lines are managed using $(r,\lot)$ policies or similar ones (like policies using cover).
The challenge of the multi-line case is that companies must decide the part of demand assigned to each line and the lot (or cover) used for the production of each item on each line.


\section{Deterministic settings}


\subsection{Problem}


The problem described by Argon Consulting considers a set $\lines$ of assembly lines producing a set $\REF$ of items over an infinite horizon.
The internal production time of item $i$ (\ie the quantity of item $i$ produced in one time unit) is $\rate_i$.
Inventory $\inventory_i$ of item $i$ must satisfy demand, known in advance.
Thus, it continuously decreases by $\demand_i<\rate_i$ units per time unit when there is no production and continuously increases by $\rate_i-\demand_i$ units per time unit when item $i$ is produced.




\section{Bibliography}

\begin{itemize}
  \item bibliographie (groupée avec celle du chapitre single line?)
  \item Flexibilité
  \item Production on a several lines
\end{itemize}




\section{Deterministic model (\tbc)}

\begin{itemize}
  \item Cas des fréquences continues avec des $\demand_r=0$ sur une seule ligne
  \item Minimisation dd'une fonction concave sur un polyhèdre
  \item Pas d'expériences numériques: Argon pas intéressé et a fait une heuristique
\end{itemize}
\esgil{Complexité du modèle}



\section{Stochastic model}

\esgil{to do: Stochastic model}

\begin{itemize}
  \item A-t-on une réduction similaire au cas déterministe? \tbc
\end{itemize}

