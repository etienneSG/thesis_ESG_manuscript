\chapter{Production on several lines}
\label{chap:lot-size:several-lines}



\cref{chap:lot-size:single-line} was dedicated to decide the optimal cycle stocks when only a single is involved.
In this chapter, we decide the optimal cycle stocks when several lines are involved in production.



\section{Motivations}
\label{sec:lot-size:several-lines:motivations}


Context is almost the same than in \cref{chap:lot-size:single-line}.
A company aims at reducing it cycle stocks (which globally helps to decrease the inventory) subject to flexibility constraint of the assemblies lines.







% Cycle stocks form the portion of inventory that varies over time due to production and demand satisfaction.
% Low cycle stocks contribute to globally decrease inventory but it requires a high flexibility of means of production.
% For mid-term decisions, flexibility is already fixed and industrial aims at deciding the values of cycle stocks of each item which minimize the global inventory.


% Many production systems are managed using $(r,\lot)$ policies or similar ones.
% This kind of policies defines for each item a level $r$ and a quantity $\lot$ such that: when inventory level of item reaches level $r$, a quantity $\lot$ is produced.
% A common variation uses cover-sizes.
% Quantity $\lot$ is replaced by a time $\cover$ called \emph{cover-size}.
% When inventory level of item $i$ reaches level $r$, quantity produced is the cumulative demand of the $\cover$ next units of time.
% Both are used by industrials (and are equivalent when demand does not depend on time).






\section{Bibliography}

\begin{itemize}
  \item bibliographie (groupée avec celle du chapitre single line?)
  \item Flexibilité
  \item Production on a several lines
\end{itemize}




\section{Deterministic model (\tbc)}

\begin{itemize}
  \item Cas des fréquences continues avec des $\demand_r=0$ sur une seule ligne
  \item Minimisation dd'une fonction concave sur un polyhèdre
  \item Pas d'expériences numériques: Argon pas intéressé et a fait une heuristique
\end{itemize}
\esgil{Complexité du modèle}



\section{Stochastic model}

\esgil{to do: Stochastic model}

\begin{itemize}
  \item A-t-on une réduction similaire au cas déterministe? \tbc
\end{itemize}

