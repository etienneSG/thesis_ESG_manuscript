\chapter{Production on several lines}
\label{chap:lot-size:several-lines}



\cref{chap:lot-size:single-line} was about to decide the optimal cycle stocks when only a single line was involved.
In this chapter, several lines are involved in production and we aim at deciding for each line the part of demand assigned and the cycle stocks for each item.



\section{Motivations}
\label{sec:lot-size:several-lines:motivations}


The context is almost the same as in \cref{chap:lot-size:single-line}.
A company aims at reducing its cycle stocks (which globally helps decrease the inventory) subject to flexibility constraints of the assembly lines.
In many applications, a company has many assembly lines and production of a single item may be done on several lines.
This kind of configuration is very common and is the consequence of multi-sourcing decisions previously made (see \cref{sec:business-context:argon:multi-sourcing} for a description of multi-sourcing issues).
These lines might be in the same plants or in different ones.
But, in both cases, it enables to increase flexibility of production and to balance the workload between lines.


As in the single-line case, assembly lines are managed using $(r,\lot)$ policies or similar ones (like policies using cover-sizes).
The challenge of the multi-line case is that the company must decide the part of the demand assigned to each line and the lot-sizes (or cover-sizes) used for the production of each item on each line.


In this chapter, we show how to compute the assignment of the demand to lines and the lot-sizes and cover-sizes minimizing inventory subject to flexibility constraints.


\section{Problem}


The problem described by Argon Consulting considers a set $\lines$ of assembly lines producing a set $\REF$ of items over an infinite horizon.
Each line is managed using the same policy than in \cref{chap:lot-size:single-line}.
The parameters for each line $\ell$ are then the same (with an index depending on the line) and we recall them for the sake of completeness.
The internal production time of item $i$ on line $\ell$ (\ie the quantity of item $i$ produced in one time unit) is $\rate_i^{\ell}$.
Inventory of item $i$ must satisfy a demand, known in advance.
The whole demand of item $i$ is modeled by a continuous decrease $\demand_i$ of the inventory, which is assumed positive.



We introduce now the decision variables $\demand_i^{\ell}$, $\cover_i$ and $t_i$ of the problem.
The part of demand for item $i$ assigned to line $\ell$ is denoted $\demand_i^{\ell}$.
It implies that the inventory of item $i$ produced by line $\ell$  continuously decreases by $\demand_i^{\ell}<\rate_i^{\ell}$ units per time unit when line $\ell$ does not produce item $i$ and continuously increases by $\rate_i^{\ell}-\demand_i^{\ell}$ units per time unit when item $i$ is produced.
If item $i$ is assigned to line $\ell$, the first time it is produced on line $\ell$ is $t_i^{\ell}$ and the production is launched for every $t_i^{\ell}+k\cover_i^{\ell}$ where $k\in\ZZ_+$ and $\cover_i^{\ell}$ is the cover-size of item $i$ for the line $\ell$.
Each production on line $\ell$ lasts $\frac{\demand_i^{\ell}}{\rate_i^{\ell}}\cover_i^{\ell}$ in order to produce exactly the demand assign to line $\ell$ for the next $\cover_i^{\ell}$ unit of time.
(Each of these quantities is not defined if item $i$ is not assigned to line $\ell$.)
Like in the single-line case, the productions of several items can be launched simultaneously.
The assignment decisions, the production decisions and the demand give the inventory of item $i$ produced by line $\ell$ which is denoted $\inventory_i^{\ell}$.



Thus, if item $i$ is assigned to line $\ell$, inventory $\inventory_i^{\ell}\bracket{t}$ generated by line $\ell$ is continuous, right and left differentiable, nonnegative and follows the dynamic
\begin{equation}\label{eq:lot-size:multi-line:motivations:dynamic}
  \dot{\inventory}_i^{\ell}\bracket{t} =
  \left\{
  \begin{array}{ll}
  \rate_i^{\ell}-\demand_i^{\ell}
  & \ds\mbox{if}\ t\in\bigcup_{k\in\ZZ_+} \left[t_i^{\ell}+k\cover_i^{\ell},t_i^{\ell}+\bracket{k+\frac{\demand_i^{\ell}}{\rate_i^{\ell}}}\cover_i^{\ell}\right),
  \\
  -\demand_i^{\ell}
  & \mbox{otherwise}.
  \end{array}
  \right.
\end{equation}


Since the whole demand must be satisfied, we have
\begin{equation}\label{eq:lot-size:multi-line:motivations:whole-demand}
  \sum_{\ell\in\lines}\demand_i^{\ell}=\demand_i\qquad\forall i\in\REF.
\end{equation}


Contrary to the single-line case, each line has a limited capacity.
In average over the infinite horizon, the percentage of time spent for production of all items assigned to line $\ell$ must be smaller than $\capacity^{\ell}<1$ which can be written as
\begin{equation}\label{eq:lot-size:multi-line:motivations:capacity}
  \sum_{i\in\REF}\frac{\demand_i^{\ell}}{\rate_i^{\ell}}\le\capacity^{\ell}\qquad\forall \ell\in\lines.
\end{equation}
Indeed, each production of item $i$ on line $\ell$ lasts $\frac{\demand_i^{\ell}}{\rate_i^{\ell}}\cover_i^{\ell}$ (in order to produce exactly the demand for the next $\cover_i$ unit of time) and then uses $\frac{\demand_i^{\ell}}{\rate_i^{\ell}}$ percents of the production time.


Like single-line case, in average over the infinite horizon, the number of setups per unit time for all items must be smaller than $\nbsetups^{\ell}$ which can be written as
\begin{equation}\label{eq:lot-size:multi-line:motivations:flexibility}
  \limsup_{\horizon\rightarrow+\infty}\ \frac{1}{\horizon} \sum_{i\in\REF^{\ell}} \left\lfloor\frac{\horizon-t_i^{\ell}}{\cover_i^{\ell}}\right\rfloor \le \nbsetups
\end{equation}
where $\REF^{\ell}$ is the set of items produced by line $\ell$.


For each item $i$, there is an initial inventory $\inventory_i\bracket{0}\in\RR_+$ given in input.


Storing one unit of item $i$ produced by line $\ell$ incurs a unit holding cost $\holding_i^{\ell}>0$ per unit time.
The objective is to find the part $\demand_i^{\ell}$ of the demand for item $i$ assigned to line $\ell$ and the cover-sizes $\cover_i^{\ell}$ which minimize the average cycle stock of every lines over infinite horizon
\begin{equation}
  \limsup_{\horizon\rightarrow+\infty}\ \frac{1}{\horizon} \sum_{\ell\in\REF} \sum_{i\in\REF} \holding_i \int_0^{\horizon}\inventory_i^{\ell}\bracket{t}dt
\end{equation}
while satisfying nonnegative inventory, constraints~\eqref{eq:lot-size:multi-line:motivations:whole-demand} to \eqref{eq:lot-size:multi-line:motivations:flexibility}.


We call this model the \emph{multi-line Economic Production Quantity model with Bounded number of Setups (the multi-line EPQ-BS)}.

Like for the single-line case, when using lot-sizes $\lot_i^{\ell}$ instead of cover-sizes, the multi-line EPQ-BS can be simply adapted using $\lot_i^{\ell}=\demand_i^{\ell}\cover_i^{\ell}$.



\section{Bibliography}



In the literature, our problem is often formulated as the decision of the assignment of production between many suppliers, the ordered quantities and the time at which they are ordered.
For example, \citet{Hong1992} propose a model with a unique item produced by many suppliers which have not the same production costs neither the same production qualities.
They aim at minimizing the holding cost subject to an upper bound on production cost and a lower bound on production quality.
Their approach consists in formulating a multiple-sourcing model and they solve a mathematical programming problem to obtain the optimal selection of suppliers and the size of the split orders.


More recently, \citet{Rosenblatt1998} propose a model considering one item and multiple suppliers.
Each order to a supplier follows the classical assumptions of the EOQ model (as described by \citet{Harris1913}) but they have a capacity constraint.
They prove that the problem is $\NP$-hard but that it can be efficiently solved with dynamic programming.
\citet{Chang2006} proposes an extension to adapt the model to real world constraints.
Other models, which also use ordering costs, are proposed by \citet{Kim2005} or \citet{Park2006}.


Another important single item model comes from \citet{Chauhan2003}.
It considers multiple suppliers delivering to one or multiple plants.
The main difference is that each ordering cost is a concave function of the ordered quantity which models the advantage of mono-sourcing the production.
They propose an heuristic for their formulation.


The most recent model which is the closest to ours is proposed by \citet{Nobil2016}.
They consider a set of lines producing a set of items.
Each line is managed using classic EPQ model (as described by \citet{Taft1918}).
However, contrary to our case, the production of one item cannot be split between two lines.
By deciding the assignment and the cover-sizes, they aim at minimizing total cost of the inventory system, setups, production, holding and disposal costs.





\section{Solving the multi-line EPQ-BS}


%\esgil{Missing: $\NP$-hardness of the model?}



We address the following alternative mathematical problem.
We are going to show that it models the multi-line EPQ-BS.
\begin{subequations}\label{eq:lot-size:multi-line:unconstrained}
  \begin{align+}
  \min\quad & \ds\sum_{i\in\REF} \frac{1}{2\nbsetups^{\ell}}\bracket{\sum_{i\in\REF}\sqrt{\holding_i^{\ell}\bracket{1-\frac{\demand_i^{\ell}}{\rate_i^{\ell}}}\demand_i^{\ell}}}^2
  \label{eq:lot-size:multi-line:unconstrained:objective}
  \\
  \st\quad  & \sum_{\ell\in\lines}\demand_i^{\ell}=\demand_i && \forall i\in\REF,
  \label{eq:lot-size:multi-line:unconstrained:flexibility}
  \\
            & \sum_{i\in\REF}\frac{\demand_i^{\ell}}{\rate_i^{\ell}}\le\capacity^{\ell} && \forall \ell\in\lines,
  \label{eq:lot-size:multi-line:unconstrained:capacity}
  \\
            & \demand_i^{\ell} \ge 0 && \forall i\in\REF.
  \label{eq:lot-size:multi-line:unconstrained:positivity}
  \end{align+}
\end{subequations}


\cref{lem:concave-objective-function} proves that the objective function is concave.
Minimization of a concave function over a polyhedron is $\NP$-hard since it contains Zero-One Linear Programming (see \citet{Raghavachari1969}) which is known to be $\NP$-hard (see \citet{Garey1979}).
However, there are many efficients algorithms to solve this kind of problems (see \citet{Tuy1964} for the first proposed algorithm and \citet{Benson1998} for a review) and it is well known that when the polyhedron is a polytope, at least one optimal solution is an extreme point (see \citet{Benson1985}).


\cref{thm:lot-size:multi-line:unconstrained:optimality} gives the formulas of the optimal cover-sizes when an optimal assignment $\bracket{\demand_i^{\ell*}}_{i\in\REF,\ell\in\lines}$ has been found.
Note that \cref{eq:lot-size:multi-line:unconstrained:optimal-cover} is well-defined.
Indeed, $\capacity^{\ell}$ is smaller than 1.
Thus, the constraint~\eqref{eq:lot-size:multi-line:unconstrained:capacity} and the definition of the cover-sizes only for the items assigned to a line ensure that $\bracket{1-\frac{\demand_i^{\ell*}}{\rate_i}}\demand_i^{\ell*}>0$.


\begin{prop}\label{lem:concave-objective-function}
The objective function
\begin{equation}
\begin{array}{rccl}
  g: & \ds\prod_{\ell\in\lines,\ i\in\REF} \sqbracket{0,\rate_i^{\ell}} & \to & \RR_+ \\
     & \bracket{\demand_i^{\ell}}_{i,\ell} & \mapsto & \ds\sum_{i\in\REF} \frac{1}{2\nbsetups^{\ell}}\bracket{\sum_{i\in\REF}\sqrt{\holding_i^{\ell}\bracket{1-\frac{\demand_i^{\ell}}{\rate_i^{\ell}}}\demand_i^{\ell}}}^2
\end{array}
\end{equation}
of Problem~\eqref{eq:lot-size:multi-line:unconstrained} is concave.
\end{prop}


\begin{proof}
Let $F: x\in\RR_+^{\REF} \mapsto \bracket{\sum_{i\in\REF} \sqrt{x_i}}^2 \in \RR_+$ be a map.
$F$ is continuous on $\RR_+^{\REF}$ and differentiable on $(\RR_+^*)^{\REF}$ and we have for all $x$ and $y$ in $(\RR_+^*)^{\REF}$
\begin{equation}
  \frac{\partial F}{\partial x_i}\bracket{x} = \sqrt{F(x)}\sqrt{x_i} \quad\forall i\in\REF,
\end{equation}
and
\begin{multline}
\left<\nabla F(y)-\nabla F(x),y-x\right>
\\
=
-\frac{1}{2}\sum_{i\in\REF}\sum_{j\in\REF}
\sqbracket
{
  \bracket{ \sqrt[4]{\frac{y_i}{y_j}}\sqrt{x_j} - \sqrt[4]{\frac{y_j}{y_i}}\sqrt{x_i} }^2
  +
  \bracket{ \sqrt[4]{\frac{x_i}{x_j}}\sqrt{y_j} - \sqrt[4]{\frac{x_j}{x_i}}\sqrt{y_i} }^2
}
\le 0
\end{multline}
Thus, $F$ is concave on $(\RR_+^*)^{\REF}$.
Since $F$ is continuous on $\RR_+^{\REF}$, it is concave on $\RR_+^{\REF}$.
For each $x$ and $x'$ in $\RR_+^{\REF}$, the inequality $x\le x'$ implies that $F(x)\le F(x')$.
Since the maps $f_i^{\ell}:\demand_i^{\ell}\in\sqbracket{0,\rate_i^{\ell}}\mapsto\holding_i^{\ell}\bracket{1-\frac{\demand_i^{\ell}}{\rate_i^{\ell}}}\demand_i^{\ell}$ are concave and since $F$ is concave and monotone (\ie $z\le z'\implies F(z)\le F(z')$), we have that
\begin{equation}
\begin{array}{rccl}
  g^{\ell}: & \sqbracket{0,\rate_1^{\ell}}\times\cdots\times\sqbracket{0,\rate_1^{\ell}} & \to     & \RR_+ \\
            & \bracket{\demand_1^{\ell},\ldots,\demand_{\REF}^{\ell}} & \mapsto & \bracket{\sum_{i\in\REF}\sqrt{\holding_i^{\ell}\bracket{1-\frac{\demand_i^{\ell}}{\rate_i}}\demand_i^{\ell}}}^2
\end{array}
\end{equation}
is concave for each line $\ell\in\lines$.
Thus the objective function $g$ of Problem~\eqref{eq:lot-size:multi-line:unconstrained} is concave as linear combination with nonnegative coefficients of concave functions.
\end{proof}


\begin{prop}\label{thm:lot-size:multi-line:unconstrained:optimality}
An optimal solution $\bracket{\demand_i^{\ell*}}_{i\in\REF,\ell\in\lines}$ of Problem~\eqref{eq:lot-size:multi-line:unconstrained} can be completed in an optimal solution of multi-line EPQ-BS by setting
\begin{equation}\label{eq:lot-size:multi-line:unconstrained:optimal-cover}
  \cover_i^{\ell*}= \frac{\sum_{j\in\REF}\sqrt{\holding_j\bracket{1-\frac{\demand_j^{\ell*}}{\rate_j}}\demand_j^{\ell*}}}{\nbsetups\sqrt{\holding_i\bracket{1-\frac{\demand_i^{\ell*}}{\rate_i}}\demand_i^{\ell*}}}
\end{equation}
when item $i$ assigned to line $\ell$.
\end{prop}


% We recall the concave monotone superposition of a convex vector-function by a monotone convex function.


% \begin{lem}[``Concave monotone superposition'']\label{lem:concave-monotone-superposition}
% Let $C$ be a convex subset of $\RR^n$.
% Let $f:x\in C \mapsto f(x) = \bracket{f_1(x),\ldots,f_k(x)}$ be vector-function on $C\subseteq\RR^n$ with concave components $f_i$, and assume that $F$ is a concave function on $\RR^k$ which is monotone, \ie such that $z\le z'$ implies that $F(z)\le F(z')$.
% Then the superposition $\phi(x) = F\bracket{f(x)}=F\bracket{f_1(x),\ldots,f_k(x)}$ is concave on $C$.
% \end{lem}


% \begin{proof}
% Let $x$ and $x'$ be two vectors of $C$ and let $\lambda$ be a real number in $(0,1)$.
% Since each $f_i$ is concave, we have
% $f\bracket{\lambda x +(1-\lambda) x'} \ge \lambda f(x) + (1-\lambda) f(x')$.
% And then,
% \begin{subequations}
% \begin{align}
%   F\bracket{f\bracket{\lambda x +(1-\lambda) x'}}
%   &\ge F\bracket{\lambda f(x) + (1-\lambda) f(x')}
%   && \mbox{(monotonicity of $F$)}
%   \\
%   &\ge \lambda F\bracket{f(x)} + (1-\lambda) F\bracket{f(x')}.
%   && \mbox{(concavity of $F$)}
% \end{align}
% \end{subequations}
% Finally $\phi\bracket{\lambda x +(1-\lambda) x'} \ge \lambda \phi(x) + (1-\lambda) \phi(x')$ and $\phi$ is concave on $C$.
% \end{proof}




\begin{proof}
Let $\bracket{\demand_i^{\ell}}_{i,\ell}$ be a feasible solution of Problem~\eqref{eq:lot-size:multi-line:unconstrained}.
We have now $\card{\lines}$ independent problems (one for each line) that correspond to the single-line case.
Consider a line $\ell\in\lines$ and the set $\REF^{\ell}$ of the items assigned to $\ell$.
According to \cref{thm:lot-size:single-line:deterministic:unconstrained:optimality}, the cover-sizes which minimize the average holding cost over infinite horizon and satisfy nonnegative inventory constraint~\eqref{eq:lot-size:multi-line:motivations:dynamic} and constraint~\eqref{eq:lot-size:multi-line:motivations:flexibility} are given by \cref{eq:lot-size:multi-line:unconstrained:optimal-cover} and the corresponding cost is equal to
\begin{equation}\label{eq:lot-size:multi-line:unconstrained:optimality:cost}
  \frac{1}{2\nbsetups}\bracket{\sum_{i\in\REF^{\ell}}\sqrt{\holding_i^{\ell}\bracket{1-\frac{\demand_i^{\ell}}{\rate_i^{\ell}}}\demand_i^{\ell}}}^2.
\end{equation}
This expression is one term of the sum in the objective function~\eqref{eq:lot-size:multi-line:unconstrained:objective} and the cover-sizes are feasible to the multi-line EPQ-BS.
Indices in this sum expressing the holding costs~\eqref{eq:lot-size:multi-line:unconstrained:optimality:cost} can be extend to $\REF$ (since assigned demand is equal to zero).
Thus, a feasible solution of Problem~\eqref{eq:lot-size:multi-line:unconstrained} can be completed in a feasible solution of multi-line EPQ-BS with same cost.


Conversely, let $\bracket{\demand_i^{\ell},\cover_i^{\ell}}_{i,\ell}$ be a feasible solution of multi-line EPQ-BS.
It is clearly feasible for Problem~\eqref{eq:lot-size:multi-line:unconstrained}.
According to \cref{thm:lot-size:single-line:deterministic:unconstrained:optimality}, for each line $\ell$, average holding cost over infinite horizon is greater than or equal to \eqref{eq:lot-size:multi-line:unconstrained:optimality:cost}.
Thus, a feasible solution of multi-line EPQ-BS can be completed in a feasible solution of Problem~\eqref{eq:lot-size:multi-line:unconstrained} whose cost is greater than or equal to \eqref{eq:lot-size:multi-line:unconstrained:optimality:cost}.
\end{proof}

