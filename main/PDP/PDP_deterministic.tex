\chapter{Deterministic CLSP-BS}
\label{chap:PDP - deterministic}


\section{Motivations}
\label{sec:PDP:Motivations}

Fixing the production level for the forthcoming period is a basic decision to be taken when managing a assembly line. Usually, a demand has to be satisfied at due dates but the limited capacity of the line prevents last minute production. On the other hand, too early productions may lead to unnecessary high inventory costs. The challenge of this kind of problems, known as {\em lot-sizing problems} in the operational research community, consists in finding a trade-off between demand satisfaction and holding costs.
When several items can be produced on a same line -- the so-called {\em multi-item} lot-sizing problem --, the capacity of the assembly line is often all the more reduced as the number of distinct items produced over the current period is high. Indeed, changing an item in production stops the line for a moment. This additional capacity reduction is usually modeled by setup costs contributing to the total cost.


Within the partnership with Argon Consulting, a non-standard feature of the problem arose. The capacity reduction due to production setups is not modeled by setup costs but instead by an explicit upper bound on the total number of items that can be produced over a period. According to Argon Consulting, many clients aim at minimizing mainly their inventory costs while keeping the number of distinct items produced over each periods below some threshold. This is essentially because, contrary to inventory costs, setup costs are hard to quantify and a maximal number of possible setups per period is easy to estimate.


To the best of our knowledge, the problem addressed in this chapter is original and such a bound on the number of distinct items produced over a period has not been considered by academics yet, with the notable exception of \cite{Rubaszewski2011} but, contrary to our problem, their bound is an overall bound for the whole horizon and they still consider setup costs.


\section{Bibliography}


Lot-sizing problems are a well-studied topic, with many variations (deterministic/stochastic, single/multi item, capacitated/uncapacitated, etc.). Recent surveys have been proposed: see \cite{Gicquel2008,quadt2008capacitated} for the deterministic version and \cite{Mula2006,Aloulou2014,Diaz-Madronero2014} for the stochastic version.

\esgil{Complete bibliography with examples of models and new features considered since the beginning.}

\cite{Pochet2006} chapter 12. Joint resource classification



\section{Model formulation}


\subsection{Most general formulation}

The assembly line produces a set $\REF$ of items over $\horizon$ periods. The number of distinct items produced over a period $t$ cannot exceed $\nbsetups_t$. There is also an upper bound $\capacity_t$ on the total period production (summed over all items) and an upper bound $\capacity_t^i$ on the production of item $i$ at period $t$. The capacity needed (in time units) to produce one unit of $i$ in period $t$ is $\rate_t^i$.

The production of item $i$ must satisfy a demand $\demand_t^i$ at the end of period $t$. When production of a item $i$ is not used to satisfy the demand, it can be stored but incurs a unit holding cost $\holding^i>0$ per period. Note that there is no setup cost, as discussed in section~\ref{sec:PDP:Motivations}. For each item $i$, there is an initial inventory $s_0^i\in\mathbb{R}_+$.

The quantity of item $i$ produced at period $t$ is denoted by $\quantity$ and the inventory level at the end of the period is denoted by $\inventory_t^i$. We also introduce a binary variable $\setup_t^i$ which takes the value 1 if the item $i$ is produced during period $t$.

The lot-sizing problem whose objective is to minimize the holding costs over the whole horizon can be written as:
\begin{subequations}\label{eq:CLSP-BS}
  \begin{align}
    \min\quad & \rlap{$\ds \sum_{t=1}^{\horizon} \sum_{i\in\REF} \holding^i \inventory_t^i$}
    \label{eq:CLSP-BS:objective}
    \\
    \st\quad & \ds \inventory_t^i = \inventory_{t-1}^i + \quantity_t^i - \demand_t^i && \forall t\in\range{\horizon},\ \forall i\in\REF,
    \label{eq:CLSP-BS:inventory-dynamic}
    \\
    & \ds \sum_{i\in\REF} \rate_t^i \quantity_t^i \le \capacity_t && \forall t\in\range{\horizon},
    \label{eq:CLSP-BS:capacity}
    \\
    & \ds \rate_t^i \quantity_t^i \le \capacity_t^i \setup_t^i && \forall t\in\range{\horizon},\ \forall i\in\REF,
    \label{eq:CLSP-BS:big-M}
    \\
    & \ds \sum_{i\in\REF} \setup_t^i \le \nbsetups_t && \forall t\in\range{\horizon},
    \label{eq:CLSP-BS:setups}
    \\
    & \ds \setup_t^i \in \crbracket{0,1} && \forall t\in\range{\horizon},\ \forall i\in\REF,
    \label{eq:CLSP-BS:boolean}
    \\
    & \ds \quantity_t^i,\ \inventory_t^i \ge 0 && \forall t\in\range{\horizon},\ \forall i\in\REF,
    \label{eq:CLSP-BS:positivity}
  \end{align}
\end{subequations}

Constraint~\eqref{eq:CLSP-BS:inventory-dynamic} is the inventory dynamic.
Constraint~\eqref{eq:CLSP-BS:capacity} prevents from exceeding the capacity of the assembly line.
Constraint~\eqref{eq:CLSP-BS:big-M} is both a ``big-M'' constraint and a capacity of the production of a single item.
Constraint~\eqref{eq:CLSP-BS:setups} limits the number of setups at each period.
Note that without loss of generality, we can suppose that $\capacity_t^i \le \capacity_t$ for each period $t$ and each item $i$.

Since this problem directly inherit the Capacity Lot-Sizing Problem (CLSP), we call this formulation \emph{Capacitated Lot-Sizing Problem with Bounded number of Setups (CLSP-BS)}. 


\medskip

In many of our applications, the capacity $\rate_t^i$ does not depend on the period and the upper bounds $\capacity_t^i$ and $\capacity_t$ and the maximal number $\nbsetups_t$ of setups does not depend on the item nor the period. Then, we normalize all quantities and replace accordingly the holding cost so that the optimization problem can be written as:

\begin{subequations}\label{eq:Normalized-CLSP-BS}
  \begin{align}
    \min\quad & \rlap{$\ds \sum_{t=1}^{\horizon} \sum_{i\in\REF} \holding^i \inventory_t^i$}
    \label{eq:Normalized-CLSP-BS:objective}
    \\
    \st\quad & \ds \inventory_t^i = \inventory_{t-1}^i + \quantity_t^i - \demand_t^i && \forall t\in\range{\horizon},\ \forall i\in\REF,
    \label{eq:Normalized-CLSP-BS:stock-dynamics}
    \\
    & \ds \sum_{i\in\REF} \quantity_t^i \le 1 && \forall t\in\range{\horizon},
    \label{eq:Normalized-CLSP-BS:capacity}
    \\
    & \ds \quantity_t^i \le \setup_t^i && \forall t\in\range{\horizon},\ \forall i\in\REF,
    \label{eq:Normalized-CLSP-BS:item-capacity}
    \\
    & \ds \sum_{i\in\REF} \setup_t^i \le \nbsetups && \forall t\in\range{\horizon},
    \label{eq:Normalized-CLSP-BS:setups}
    \\
    & \ds \setup_t^i \in \crbracket{0,1} && \forall t\in\range{\horizon},\ \forall i\in\REF,
    \label{eq:Normalized-CLSP-BS:boolean}
    \\
    & \ds \quantity_t^i,\ \inventory_t^i \ge 0 && \forall t\in\range{\horizon},\ \forall i\in\REF,
    \label{eq:Normalized-CLSP-BS:positivity}
  \end{align}
\end{subequations}

This version is called the \emph{Normalized Capacitated Lot-Sizing Problem with Bounded number of Setups (Normalized CLSP-BS)}. This formulation capture the essential part and the difficulty of the problem which is the limited flexibility of the assembly line. Most results are true for the CLPS-BS, but we will give counterexample when result only stand for the Normalized CLSP-BS.


\section{Theoretical results}


\subsection{NP-completeness}


\subsection{Relaxations}

\subsubsection{Continuous relaxation}

\begin{itemize}
  \item does not depend on N (only for normalized version)
\end{itemize}

\subsubsection{Lagrangian relaxations}

\begin{itemize}
  \item 2 Lagrangian relaxations
\end{itemize}


\subsection{Valid inequality}


\subsection{Extended formulations}

\begin{itemize}
  \item 2 extended formulations
  \item bound equalities
\end{itemize}


\subsection{Complexité paramétrée: résultats à $\REF$ fixé}

\esgil{TODO}

\begin{itemize}
  \item polynomial à R fixé par programmation dynamique
\end{itemize}
